\setbeamertemplate{navigation symbols}{} %% 右下のアイコンを消す
% \AtBeginDvi{\special{pdf:tounicode 90ms-RKSJ-UCS2}} %% しおりが文字化けしないように
\AtBeginDvi{\special{pdf:tounicode EUC-UCS2}}
\usepackage[utf8]{inputenc}
\usepackage[absolute,overlay]{textpos}
\usepackage{fancybox}
\usepackage{tikz}
\usetikzlibrary{trees}
\usepackage{pxpgfmark}
\usetikzlibrary{arrows}
\usetikzlibrary{shapes}
\usetikzlibrary{positioning}
\newcommand{\highlight}[2][yellow]{\tikz[baseline=(x.base)]{\node[rectangle,rounded corners,fill=#1!10](x){#2};}}
\newcommand{\highlightcap}[3][yellow]{\tikz[baseline=(x.base)]{\node[rectangle,rounded corners,fill=#1!10](x){#2} node[below of=x, color=#1]{#3};}}

\usepackage{color}
\tikzset{markplace/.style={rectangle callout,fill=red!10,draw=none,callout absolute pointer={#1}, at={#1},above=1cm}}

\definecolor{light-gray}{gray}{0.80}
\usepackage{listings}
\lstdefinestyle{customc}{
  belowcaptionskip=1\baselineskip,
  breaklines=true,
  frame=L,
  xleftmargin=\parindent,
  language=C,
  showstringspaces=false,
  basicstyle=\fontsize{5}{5}\ttfamily,
  % basicstyle=\footnotesize\ttfamily,
  keywordstyle=\bfseries\color{green!40!black},
  commentstyle=\itshape\color{purple!40!black},
  identifierstyle=\color{black},
  stringstyle=\color{orange},
}
\lstdefinestyle{largec}{
  belowcaptionskip=1\baselineskip,
  breaklines=true,
  frame=L,
  xleftmargin=\parindent,
  language=C,
  showstringspaces=false,
  basicstyle=\fontsize{8}{8}\ttfamily,
  % basicstyle=\footnotesize\ttfamily,
  keywordstyle=\bfseries\color{green!40!black},
  commentstyle=\itshape\color{purple!40!black},
  identifierstyle=\color{black},
  stringstyle=\color{orange},
}
\usepackage{setspace} % setspaceパッケージのインクルード
\setstretch{1.5}
\usepackage{multicol}
\usepackage{cancel}
\renewcommand{\CancelColor}{\color{red}}
\usepackage{ulem}

\usepackage{graphicx}
\newlength{\mytotalwidth}
\mytotalwidth=\dimexpr\linewidth-5mm
\newlength{\mycolumnwidth}
\mycolumnwidth=\dimexpr\mytotalwidth-5mm

%\useoutertheme{shadow}                 %% 箱に影をつける
\usefonttheme{professionalfonts}       %% 数式の文字を通常の LaTeX と同じにする
%\setbeamercovered{transparent}         %% 消えている文字をうっすらと表示する
\setbeamertemplate{theorems}[numbered]  %% 定理に番号をつける

% \usetheme{CambridgeUS}         %% theme の選択
% \usetheme{Boadilla}           %% Beamer のディレクトリの中の
\usetheme{Madrid}             %% beamerthemeCambridgeUS.sty を指定
% \usetheme{Antibes}            %% 色々と試してみるといいだろう
%\usetheme{Montpellier}        %% サンプルが beamer\doc に色々とある。
%\usetheme{Berkeley}
%\usetheme{Goettingen}
%\usetheme{Singapore}
%\usetheme{Szeged}

% \usecolortheme{rose}          %% colortheme を選ぶと色使いが変わる
% \usecolortheme{albatross}
\usecolortheme{beaver}

\title[4th]{輪講会第4回}
\author[Ryo Yoda, Shota Tsuji]{j114126 依田凌\\ j114081 辻祥太}
\institute[Kogakuin]{Kogakuin Univ}
\subject{Computer Archtecutre}
