\documentclass[11pt,a4paper]{jsarticle}
%
\usepackage{amsmath,amssymb}
\usepackage{bm}
\usepackage{graphicx}
\usepackage{ascmac}
%
\setlength{\textwidth}{\fullwidth}
\setlength{\textheight}{39\baselineskip}

\begin{document}
\section{range of 2016}
\begin{enumerate}
  \item 線形代数-行列
  \item 線形代数-固有値と固有ベクトル
  \item 微分・積分学-定積分
  \item 微分・積分学-偏微分-オイラー・ラグランジェ方程式
  \item (偏)微分方程式-1階偏微分方程式
  \item 確率・統計-順列・組み合わせ
\end{enumerate}

\section{range of 2015}
\begin{enumerate}
  \item 線形代数-固有値と固有ベクトル
  \item (偏)微分方程式-行列
  \item 楕円の式
  \item 微分方程式-おそらく非線形微分方程式?
  \item 確率・統計-確率-平均,分散(,標準偏差,積率)-確率関数
\end{enumerate}

\section{range of 2014}
\begin{enumerate}
  \item 線形代数-固有値と固有ベクトル
  \item 線形代数-2次形式
  \item 微分・積分学-定積分
  \item 正規直交条件
  \item 確率・統計-確率-平均,分散(,標準偏差,積率)
  \item 確率変数列と極限
\end{enumerate}

\section{range of 2013}
\begin{enumerate}
  \item 線形代数-固有値と固有ベクトル
  \item 行列-ランク
  \item 常微分方程式
  \item 定積分
  \item 確率・統計-順列・組み合わせ
  \item 微分・積分学-数列と級数
\end{enumerate}

\section{range of 2012}
\begin{enumerate}
  \item 線形代数-固有値と固有ベクトル
  \item 極限値
  \item 微分・積分学-曲面積
  \item 確率・統計-順列・組み合わせ
  \item 確率・統計-期待値
\end{enumerate}

\end{document}
