\documentclass[10pt, a4paper]{jsarticle}
\usepackage[dvipdfmx, hiresbb]{graphicx}
\usepackage{comment}
\usepackage[top=20truemm, bottom=25truemm, left=15truemm, right=15truemm]{geometry}

\title{教育と社会 最終レポート \\ レポートテーマ 子供の貧困}
\author{j114081 辻祥太}

\begin{document}
  \maketitle
  \newpage

  貧困の原因は,物事を行なう際に支払うお金がないことである.現代では,生活するにも,学校に行くにも,就職活動をする際でもお金が必要となっている.何か行動をしようとした際に支払うお金が無ければ,その行動に制限がかかる世の中である.また貧困は,その経済的困難に付随して起こっている,健康面の問題や教育と進学に関する問題,虐待といじめの問題と関連づけられて捉えられる.以下では,教育と進学に関する問題に絞って述べていくこととする.

  私は,子供の貧困を解決するには大学進学のメリットの認識や奨学金についての理解を広める活動が必要だと考える.現代の日本にで貧困という立場にある人が大学に進学するということは,平均的な数値と比較すると,とても困難になっていることが分かる.家庭状況別の大学進学率に関するデータでは,高校全日制・定時制全体(14年3月卒)では76.2\%なのに対し,生活保護世帯31.7\%,児童養護施設22.6\%,ひとり親家庭41.6\%となっている.進学率の低さには,3つの要因があると考える.1つ目は,「奨学金について詳しく知らない・知れないがために,その費用の高さゆえに進学することを諦める」ということである.奨学金について持っている情報が少ないために,「大学は費用が高い」という部分だけを見て諦めてしまうことや,「奨学金が借りられてもいずれは借金になる」という早合点に陥ってしまっていることも多々あるだろう.2つ目は,生活保護に関するものである.生活保護を受けている家庭において子供を大学進学させる場合,世帯分離が必要になり,これにより家庭全体での生活保護受給料は数万円減少し,加えて進学する当人は別途健康保険に加入する必要が出てくる.このようないくつもの障壁に阻まれることとなる.3つ目は,貧困とされる家庭での大学進学についての認識に関するものである.現在,上の話にあるように大学費用の支援は生活保護を受ける家庭に対しては支給されない.だが,一昔前までは,高校にかかる費用すら支給されていなかったのである.1970年に世帯保護の中で高校に行くことが認められ,高校に通うための費用については2005年から生活保護として支給されるようになった.時代の変化を背景に,高校生の年齢で働くことを求めるよりも,就学して自立に向けた力を養うほうがよいという判断をしたらしい. この事例から,ベースライン*の向上(すなわち日本人の大学進学に対する認識の改善)を行なうことが必要であると考える.%貧困家庭では,親が「大学は学費が高いから行けない」と思い込んでいたり,すぐに収入を得ることを求めたり,子供の教育や進路に無関心だったりすることも少なくない.これに関しては,実体験がある.私の父親は,正規雇用になるまでアルバイトや派遣として仕事を変えてきていた.そのせいか,大学1年生の私に対し,他の選択肢を提示すること無く「アルバイトはしたほうがよい」と強く薦めてくる.このように教育内容に対して関心が低いのは,私の家系では大学進学をしたのは私が初めてなので,その意義について身を持って理解している人がいないからであろうといえる.%また祖母から,祖母の兄にあたる人が成績優秀であったにも関わらず金銭的な問題から大学進学を諦めたという話を聞いたこともある. %  教育に対する関心が低くなる,無関心になるということの要因は,「大学進学に掛かる多大な費用ゆえの諦め」と「大学に行くメリットを具体的に把握できていないこと」であると考える.もちろん,性急な問題が目の前にあるという理由もあると思うが,進学するかしないかという範囲で観点から話を進めていくため,このような場合は除くものとする.

%  より多岐にわたる情報の提示ができていないと考えられる.
%  進路ごとの学費,奨学金,学生の生活費といった経済面についての情報提供や相談できる機関が乏しいこと.https://yomidr.yomiuri.co.jp/article/20160304-oytet50010/

  この3つの要因を解消する策として,国や地域・学校からの奨学金(とくに大学)についての説明会を全国的なものとしてさらに普及させることが考えられる.これは,貧困という立場にある人たちに,ただ単に「大学は費用がとても掛かるから行けない」と思ってもらいたくないからである.具体的にいうと,全国的説明会に各大学の説明ブースを併設するという活動が挙げられる.すでに国や地域からの奨学金についての全国的な説明会(自力進学フェア)は存在するが,大学の奨学金やその大学の特徴を伝えるような催しは少ないように感じられる. そこで,高校生ひとりでは集めるのが大変であろう奨学金の情報を公開し,少しでも進学希望を現実にしてもらうことがこの活動を行なう目的である.自分自身の取り組みとしては,地方の説明会に参加し,詩文の大学の奨学金制度や自分が通うからこそ進められる大学の長所について伝えていくことができる.またこの活動は,一般の認識として奨学金についての一定以上の理解が得られ,ひいては生活保護を受ける家庭に置いても大学進学を考えられるような社会になることを見据えたものである.

  以上のことから,奨学金についての説明会を全国的なものとしてさらに普及させることは,地道であるが長い目で見ると効果的な解決策であると考えられる.\\ \\
  改善点と工夫 \\
  前回書いたレポートでは,「貧困になっている人に対して,情報提供をする」という提案をしたが,抽象的であった.また,データもなく,自分の想像で述べていた.そこで今回は,データと事実にもとづき意見を述べるように意識した.また,この問題をひとごとのように捉えて考えるのが嫌だったので,「貧困層の人々」のように表現するのではなく,「貧困の立場にある人」のようにした. \\ \\
  コメント \\
  この授業を通して,現代で実際に起きている問題について考えることができました.この授業を取っていなかったら知らずに人生を送っていたかもしれません.少し人として何か成長できたと思います.

  \begin{thebibliography}{9}

  \end{thebibliography}

  \end{document}



\begin{comment}
  話の流れ
  提起(どうしてそれが課題になるのか・課題として扱われるのか)
  貧困の原因は,物事を行なう際に支払うお金がないことである.現代では,生活するにも,学校に行くにも,就職活動をする際でもお金が必要となっている.何か行動をしようとした際に支払うお金が無ければ,その行動に制限がかかる世の中である.また貧困は,その経済的困難に付随して起こっている,健康面の問題や教育と進学に関する問題,虐待といじめの問題と関連づけられて捉えられる.以下では,教育と進学に関する問題に絞って述べていくこととする.

  子供の貧困を解決するには,大学進学のメリットの認識や奨学金についての理解を広める活動が必要だと考える.


  課題(どのようなことが要因・原因なのか)
  (しかし,)現代の日本にで貧困という立場にある人が大学に進学するということは,平均的な数値と比較すると,とても困難になっていることが分かる.
進学率*としては,高校全日制・定時制全体(14年3月卒)では76.2%なのに対し,生活保護世帯31.7%,児童養護施設22.6%,ひとり親家庭41.6%となっている.進学率の低さには,3つの要因があると考える.

  高くて進学することを諦める.奨学金について詳しく知らない・知れない.
  奨学金について持っている情報が少ないために,「大学は費用が高い」という部分だけを見て諦めてしまうことや,「奨学金が借りられてもいずれは借金になる」という早合点に陥ってしまっていることも多々あるだろう.
  生活保護を受けている家庭において子供を大学進学させる場合,世帯分離が必要になり,これにより家庭全体での生活保護受給料は数万円減少し,加えて進学する当人は別途健康保険に加入する必要が出てくる.このようないくつもの障壁に阻まれることとなる.
  貧困とされる家庭での大学進学についての認識(より一般的なものへ)
  ベースライン*の向上(すなわち日本人の大学進学に対する認識の改善)を行なうこと.1970年に世帯保護の中で高校に行くことが認めらた.高校に通うための費用は,2005年から生活保護として支給されるようになった.時代の変化を背景に,(高校生の年齢では)働くことを求めるよりも,就学して自立に向けた力を養うほうがよいという判断になってきた.  貧困家庭では,親が「大学は学費が高いから行けない」と思い込んでいたり,すぐに収入を得ることを求めたり,子供の教育や進路に無関心だったりすることも少なくない.これに関しては,実体験がある.私の父親は,正規雇用になるまでアルバイトや派遣として仕事を変えてきていた.そのせいか,大学1年生の私に対し,他の選択肢を提示すること無く「アルバイトはしたほうがよい」と強く薦めてくる.このように教育内容に対して関心が低いのは,私の家系では大学進学をしたのは私が初めてなので,その意義について身を持って理解している人がいないからであろうといえる.%また祖母から,祖母の兄にあたる人が成績優秀であったにも関わらず金銭的な問題から大学進学を諦めたという話を聞いたこともある.
  教育に対する関心が低くなる,無関心になるということの要因は,「大学進学に掛かる多大な費用ゆえの諦め」と「大学に行くメリットを具体的に把握できていないこと」であると考える.もちろん,性急な問題が目の前にあるという理由もあると思うが,進学するかしないかという範囲で観点から話を進めていくため,このような場合は除くものとする.

  より多岐にわたる情報の提示ができていないと考えられる.
  進路ごとの学費,奨学金,学生の生活費といった経済面についての情報提供や相談できる機関が乏しいこと.https://yomidr.yomiuri.co.jp/article/20160304-oytet50010/

  これに対する解決策として,国や地域・学校からの奨学金(とくに大学)についての説明会を普及させることが考えられる.これは,貧困という立場に(立っている)ある人たちに,ただ単に「大学は費用がとても掛かるから行けない」と思ってもらいたくないからである.まず大学(さらに短大・高専・専門学校)まで進学するメリットを端的に述べると,正社員として定着する割合が高くなるということである.*
  2006年2月時点での調査によると,この割合は,「大学・大学院卒」「短大・高専卒」「専門卒」「高卒」の順になっている.「大学・大学院卒」と「高卒」との割合には,男女ともに約30\%の差がある.http://www.mext.go.jp/b_menu/hakusho/html/hpab201001/detail/1312226.htm


  解決策(どのようにしたらその課題に対処できるか・前に進めていけるか)とその解決策の特徴
  奨学金の説明会を全国的なものとしてさらに普及させる
  すでに国や地域からの奨学金についての全国的な説明会は存在する.(自力進学フェア)http://shinronavi.com/column/jiriki/scholarship07.php
  また,各地方ごとの主要大学説明会のようなものも存在する.ただ専門学校などになると,各地方での説明会は無いように感じられ,首都圏での説明会やオープンキャンパスに絞られる.
  大学生として,説明会に参加する.

  結論(上記のことから言えること・自分の意見の再提示)




  貧困の原因
  経済的困難を抱えている(生活で手一杯など)
  お金が稼げない(非正規雇用・転職つづき) <- 学歴を見られる(高卒ではきつい?)
  高卒の段階では選択できる職種に大きな制限がかかる.*
  片親,不慮の事故による親の怪我・死亡

  貧困によって引き起こされる問題

  「貧困になっている人に対して,情報提供をする」というのは具体的ではない.
  まず,提供される情報としては何があるか.-> もの(服や生活用品などの物理的支援,施設(住居など)の支援,これらを仲介するものとしてのお金の支援

\begin{enumerate}
    \item 授業で扱った内容を参考にしながら,子供の貧困問題の捉え方について自分なりに整理を行なったうえで,特に重要である(軸になる)と考えられる取り組みを具体的に示し,そう考える理由を説明せよ.その際,その取り組みの実行にあたって自分自身のにはどのような行動が可能であるかということを必ず盛り込むこと.
    \item 自分が以前書いたレポートについて,どこに改善点を見出し,実際にどのような工夫をして今回のレポートを作成したかを説明せよ.
    \item 授業全体やレポートなどについてのコメント.
  \end{enumerate}

  やることリスト
  \begin{itemize}
    \item 子供の貧困問題とはどのようなものなのか(何からできているのか)を考える.また,どのように捉えられるか.
    \item 自分が大事だと思う取り組みを上げる.どうして大事だと思ったかの理由もつける.
    \item その取り組みの中で,自分にもできること・参加していけることは何かを示す.
  \end{itemize}

\end{comment}
