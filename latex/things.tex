\documentclass[dvipdfmx]{beamer}
%\documentclass[dvipdfmx,usenames]{beamer}
\usepackage{graphicx}
\usepackage{color}
\usetheme{Madrid}
\useinnertheme{circles}
\useoutertheme{default}
%\usepackage{beamerthemeshadow}
\usepackage{tikz}
\usepackage[varg]{txfonts}
\usepackage{booktabs}
\renewcommand{\kanjifamilydefault}{\gtdefault}
% 隠しているアイテムを半透明で表示
\setbeamercovered{transparent}
% ナビゲーションの非表示
\setbeamertemplate{navigation symbols}{}
\everymath{\displaystyle}
%各節の始まりで発表の流れを示す
\AtBeginSection[]
{\begin{frame}{発表の流れ}
  \tableofcontents[currentsection]
\end{frame}}
%ページ番号の表示
\setbeamertemplate{footline}[frame number]
% しおりをつくる
\usepackage{hyperref}
%しおりをつくるsectionの深さや,目次のリンクの色などを指定
%\usepackage[usenames]{color}
%\usepackage[bookmarksopenlevel=2]{hyperref}
%しおりを日本語対応にする
\usepackage{pxjahyper}
%amsmathとhyperrefとの互換性
\usepackage{amsmath}
\let\equation\gather
\let\endequation\endgather

\setbeamertemplate{enumerate item}[circle]

%ページ数カウント
\newcommand{\backupbegin}{
  \newcounter{framenumberappendix}
  \setcounter{framenumberappendix}{\value{framenumber}}
}
\newcommand{\backupend}{
  \addtocounter{framenumberappendix}{-\value{framenumber}}
  \addtocounter{framenumber}{\value{framenumberappendix}}
}


%\usepackage{emath}
%\usepackage{emathMw}
\usepackage{subcaption}

\title{毎日使ってるもの大集合}
\institute{工学院大学}
\author{情報学部コンピュータ科 j114081 \\ 辻 祥太}
\date{}

\begin{document}

\section*{はじめ}
\begin{frame}
  \titlepage
\end{frame}

\begin{frame}[plain]
  \frametitle{Contents}
  \tableofcontents
\end{frame}

\begin{frame}
  \frametitle{この資料の目的}
  やりたいこと
  \begin{itemize}
    \item どれくらいのものを使っているのか・持っているのかを知る
    \item 身の回りにあるものの整理につながる
    \item どこまですっきりさせられるかを考えてみる
  \end{itemize}
\end{frame}

\section{せっけん・水まわり}
\begin{frame}
  \frametitle{}
  \begin{itemize}
    \item 手書きの数字0\textasciitilde9
    \item サイズは28*28
    \item Trainデータ60000枚,Testデータ10000枚
  \end{itemize}
  \begin{figure}
    \centering
    \begin{minipage}{0.4\columnwidth}
      \centering
      \includegraphics[width=\columnwidth]{./rep_pic/rep_image0.png}
    \end{minipage}
    \begin{minipage}{0.4\columnwidth}
      \centering
      \includegraphics[width=\columnwidth]{./rep_pic/rep_image1.png}
    \end{minipage}
  \end{figure}
\end{frame}

\section{kNN}
\begin{frame}
  \frametitle{kNN}
  kNN法のアルゴリズム
  \begin{enumerate}
    \item \structure{kの値}を決定する
    \item 推定したい一点の\structure{近傍にあるデータ}をk個選択する
    \item k個のデータのなかで\structure{多数決をとる}
    \item 最多となるラベル(カテゴリ)を,推定ラベルとする
  \end{enumerate}
  \begin{center}
    \begin{figure}
      \includegraphics[width=0.6\paperwidth]{./rep_pic/example_knn.png}
    \end{figure}
  \end{center}
\end{frame}

\begin{frame}
  \frametitle{kNN法}
  近傍点の数kの決め方
  \begin{itemize}
    \item kが小さい $ \Rightarrow $ ノイズに弱い
    \item kが大きい $ \Rightarrow $ 他クラスのデータも含みやすい
  \end{itemize}
%  \begin{minipage}{0.4\textwidth}
%    \centering
%    \includegraphics[width=0.4\textwidth]{./example_knn2.png}
%  \end{minipage}
%  \begin{minipage}{0.4\textwidth}
%    \centering
%    \includegraphics[width=0.4\textwidth]{./example_knn3.png}
%  \end{minipage}
  \begin{center}
%    \centering
    \begin{minipage}{0.45\columnwidth}
      \centering
      \includegraphics[width=\columnwidth]{./rep_pic/example_knn3.png}
    \end{minipage}
    \begin{minipage}{0.45\columnwidth}
      \centering
      \includegraphics[width=\columnwidth]{./rep_pic/example_knn2.png}
    \end{minipage}
  \end{center}
\end{frame}

\begin{frame}
  \frametitle{kNN法}
  近傍の点をk個取ってくる \\
  今回は\structure{ユークリッド距離}
  \begin{equation}
    l_{p2p1} = \sqrt{\left( \structure{p2_{x} - p1_{x}} \right)^{2} + \left( \alert{p2_{y} - p1_{y}} \right)^{2}}
  \end{equation}
  \begin{figure}
    %\includegraphics[width=0.6\paperwidth]{./example_euclidian.png}
    \includegraphics[width=0.6\paperwidth]{./rep_pic/example_euclidian.png}
  \end{figure}
\end{frame}

\begin{frame}
  \frametitle{kNN法}
  近傍の点をk個取ってくる \\
  これができれば,あとは多数決を取るのみ \\
  Code - Python
  \begin{figure}
    \includegraphics[width=0.6\paperwidth]{./rep_pic/rep_getNeighbor.png}
  \end{figure}
\end{frame}

\begin{frame}
  \frametitle{kNN法}
  結果(k = 3, 5)
  \begin{center}
  \begin{table}
    \begin{tabular}{ c r } \toprule
      kの数 & 正答率 \\ \midrule
      3 & 77.32\% \\
      5 & 79.97\% \\ \bottomrule
    \end{tabular}
  \end{table}
  \end{center}
\end{frame}

\section{PCA}
\begin{frame}
  \frametitle{主成分分析(PCA)}
  主成分分析とは
  \begin{itemize}
    \item 次元数の多いデータを,情報量ができるだけ多くなるようなかたちで次元数を落とす手法
    \item 次元の圧縮によりデータ全体が小さくなるため,扱いやすくなるという利点
  \end{itemize}
\end{frame}

\begin{frame}
  \frametitle{主成分分析(PCA)}
   主成分分析のポイント
   \begin{block}{固有値と標準偏差}
     主成分(固有ベクトル,すなわち軸)ひとつにつき,ひとつ固有値がある.この固有値の平方根は標準偏差となる.
     \begin{equation}
       \sqrt{\lambda_{i}} = \sqrt{\frac{1}{n} \textstyle\sum_{k=1}^{n}\left(a_{k}-\mu\right)^2}
     \end{equation}
     ただし,\[ \mu = \frac{1}{n} \textstyle\sum_{k=1}^{n}a_{k} \]
%      \begin{center}
%        $ \mu = \frac{1}{n} \textstyle\sum_{k=1}^{n}a_{k} $
%      \end{center}
    \end{block}
\end{frame}

\begin{frame}
  \frametitle{主成分分析(PCA)}
   主成分分析のポイント \\
   主成分(寄与率の高いもの上から2つ)
   \begin{figure}
     \centering
     \begin{subfigure}{0.4\paperwidth}
       \centering
       \includegraphics[width=\columnwidth]{./rep_pic/EigenVector000.png}
       \caption{第一主成分(寄与率9.7\%)}
     \end{subfigure}
     \begin{subfigure}{0.4\paperwidth}
       \centering
       \includegraphics[width=\columnwidth]{./rep_pic/EigenVector001.png}
       \caption{第二主成分(寄与率7.1\%)}
     \end{subfigure}
%     \begin{subfigure}{0.3\paperwidth}
%       \centering
%       \includegraphics[width=\columnwidth]{./rep_pic/EigenVector001.png}
%       \caption{第二主成分(寄与率7.1\%)}
%     \end{subfigure}
   \end{figure}


\end{frame}

\begin{frame}
  \frametitle{主成分分析(PCA)}
   主成分分析のポイント
   \begin{block}{寄与率}
     全体の固有値に対するひとつの固有値の大きさ \\
     すなわち,全体の情報量に対して主成分ひとつが占める割合
     \begin{equation}
       r_{i} = \frac{\lambda_{i}}{ \textstyle\sum_{k=1}^{n}\lambda_{k}}
     \end{equation}
   \end{block}
   \begin{block}{累積寄与率}
     全体の固有値に対するm番目までの固有値の大きさの和 \\
     とった主成分がカバーできている情報量 $\left(m \le n\right)$
     \begin{equation}
       r_{i} = \frac{\lambda_{1}+\lambda_{2}+\cdots+\lambda_{m}}{ \lambda_{1}+\lambda_{2}+\cdots+\lambda_{n}}
     \end{equation}
   \end{block}
\end{frame}

\begin{frame}
  \frametitle{主成分分析(PCA)}
  全体の情報量のうち90\%くらいを保持しておきたい \\
  $ \Rightarrow $ 主成分およそ100個 \\
  50個ほどでも80\%に至っている
  \begin{figure}
    \centering
       \includegraphics[width=\columnwidth]{./rep_pic/cumsum_explained_D100.png}
       \caption{累積寄与率91\%}
  \end{figure}

\end{frame}

\begin{frame}
  \frametitle{PCA + kNN法}
  結果(k = 3, 5)
%  \begin{center}
%  \begin{table}
%    \begin{tabular}{ c r } \toprule
%      %kの数 & 正答率 & 経過時間 \\ \midrule
%      kの数 & 正答率  \\ \midrule
%      3 & 92.53\% \\
%      5 & 92.85\% \\
%      11 & \\ \bottomrule
%    \end{tabular}
%  \end{table}
%  \end{center}
  \begin{table}
    \centering
    \begin{minipage}{0.3\columnwidth}
      \caption{次元D= \structure{10}}
      \centering
      \begin{tabular}{c r } \toprule
      kの数 & 正答率  \\ \midrule
      3 & 92.53\% \\
      5 & 92.85\% \\ \bottomrule
      \end{tabular}
    \end{minipage}
    \begin{minipage}{0.3\columnwidth}
      \caption{次元D= \structure{50}}
      \centering
      \begin{tabular}{c r } \toprule
      kの数 & 正答率  \\ \midrule
      3 & 97.57\% \\
      5 & 97.53\% \\ \bottomrule
      \end{tabular}
    \end{minipage}
    \begin{minipage}{0.3\columnwidth}
      \caption{次元D= \structure{100}}
      \centering
      \begin{tabular}{c r } \toprule
      kの数 & 正答率  \\ \midrule
      3 & 97.36\% \\
      5 & 97.38\% \\ \bottomrule
      \end{tabular}
    \end{minipage}
  \end{table}

\end{frame}

\begin{frame}
  \frametitle{まとめ}
  \begin{itemize}
    \item 主成分分析を掛けた後の方が正答率が上がった.
    \item 近傍点の個数は多ければ良いというわけではない.
  \end{itemize}
\end{frame}

\appendix
\backupbegin

\begin{frame}
  \begin{center}
    \Huge Appendixes
  \end{center}
\end{frame}

\begin{frame}
  \frametitle{Usage of Programs}
  \begin{enumerate}
    \item 実行するディレクトリ(/a とする)に全てのプログラムを置く.
    \item 現在のディレクトリ配下にdataディレクトリを作成する(/a/data).
    \item dataディレクトリの中にMNISTの圧縮ファイルを置き,解凍しておく.
    \item convertBinary.pyは一度だけ実行する.
    \item pcaMNIST.pyとkNN.pyの実行には引数が必要となる. \\
    python pcaMNIST 100 $ \Leftarrow $ 主成分の個数を指定する \\
    python kNN.py 5 100 $ \Leftarrow $ 第一引数は近傍点の数, 第二引数は主成分の個数
    python kNN.py 5 0 $ \Leftarrow $ 第二引数に0を指定すると,主成分分析前のデータにkNNを行なう.
  \end{enumerate}

\end{frame}

\begin{frame}
  \frametitle{参考文献}
  \begin{thebibliography}{9}
    \bibitem{pca_start}
    統計学研究所 www.statistics.co.jp/reference/software\_R/statR\_9\_principal.pdf
    \bibitem{knn}
    http://machinelearningmastery.com/tutorial-to-implement-k-nearest-neighbors-in-python-from-scratch/
  \end{thebibliography}
\end{frame}

\backupend
\end{document}
