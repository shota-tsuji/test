\documentclass[11pt, a4paper]{jsarticle}
\usepackage[dvipdfmx, hiresbb]{graphicx} %Bring Images[Driver, Exactly]
\usepackage{here}
\usepackage{comment}
\usepackage{array, booktabs} %Setting for Table
\usepackage{float} %Locating Figure and Table
\usepackage[top=20truemm, bottom=25truemm, left=15truemm, right=15truemm]{geometry} %Setting Page Balance
\usepackage{listings} %For Showing Source Code
\usepackage{xcolor} %Using Color ->Source Code
\renewcommand{\lstlistingname}{ソースコード} %Caption for Soure Code
\lstset{
	language= C,
	numbers= left,		numbersep= 10pt,		numberstyle= \ttfamily,
	breaklines= ture,	breakindent= 20pt,
	frame= shadowbox,	framesep= 3pt,	rulesep= 4pt,
	stepnumber= 1,
	backgroundcolor= \color{white},
	rulesepcolor= \color{cyan},
	basicstyle= \ttfamily, columns= [1][fullflexible],
	tabsize= 4,
	label= }
\usepackage{ascmac}
%\usepackage{latexsym}
\setlength{\textwidth}{\fullwidth}
\setlength{\textheight}{39\baselineskip}


\begin{document}
\tableofcontents %Making Table of Contents
\newpage

\section{April/2017}
\begin{itembox}[r]{Apr/13}
	pythonでGFlopsの描画をした
\end{itembox}

\begin{itembox}[r]{Apr/14}
	Seminar発表スライドを作成した.\\
	つつじの駅のまわりを探索した.\\
	Nehalemのアーキテクチャを少し見た.\\
	国領駅まで走った.
\end{itembox}

\begin{itembox}[r]{Apr/15}
	歯医者さんに行った.\\
	調布駅まで走った.\\
	壁に掛かっている上着を棚の中にしまった.
\end{itembox}

\begin{itembox}[r]{Apr/16}
	randomの使い方を復習した.\\
	海外のR18の動画を初めて見た.\\
	いなげやから多摩川方面まで走った.\\
	6時間(連続)睡眠チャレンジをスタートさせた.(1日目)
\end{itembox}

\begin{itembox}[r]{Apr/17}
	HPC研究会に参加した.\\
	東大柏キャンパスまで行ってきた.\\
	Seminar発表スライドを急いで作った.(寝る時間遅くなった)
\end{itembox}

\begin{itembox}[r]{Apr/18}
	ひたすらスライドを作った.\\
	LaTexのコンパイル通らなくて,間に合わなかった.(提出遅れた)\\
	shscriptも,指定正しくできているか怪しかった.
\end{itembox}

\begin{itembox}[r]{Apr/19}
	Seminar発表した.\\
	文字サイズの改善をはかる必要があった.\\
	行列行列積において,ブロック化を試してみた.
\end{itembox}

\begin{itembox}[r]{Apr/20}
	行列ベクトル積を,mpiを使って計算した.\\
	対称行列の固有値について少し知った.(あいまい)\\
	春・夏服を見に行った.
\end{itembox}

\begin{itembox}[r]{Apr/21}
	お昼くらいからあまり頭が回らなかった.お昼ごはん食べ過ぎたかもしれない.\\
	春物の服を見に行った.しかし,あまりピンとくるものが見つけられなかった.
\end{itembox}

\begin{itembox}[r]{Apr/22}
	掃除機を久しぶりに掛けた.\\
	ためていた本を,縛って捨てられた.\\
	歯医者さんに行った.\\
	高野さんが寮に遊びにきた.\\
	長岡の花火はおすすめらしい.
\end{itembox}

\begin{itembox}[r]{Apr/23}
	自転車を修理にだした.パンクを直してもらった.\\
\end{itembox}

\begin{itembox}[r]{Apr/24}
	行列行列積について,ブロック化・ループ交換と,転置を試せた.\\
	菱沼さんは,論文を書いた内容について,「自分のものにしていた」らしい.\\
\end{itembox}

\begin{itembox}[r]{Apr/25}
	ヘネパタ輪公開で,PC相対というものを知った.\\
	ゼミのためのスライドの作成が間に合わなかった.原因は,土日に全く取り組まなかったことであろう.
\end{itembox}

\begin{itembox}[r]{Apr/26}
	Graphのプログラミングコンテスト初級問題を解いた.\\
	藤井先生と,大学院(つくば)について\&野村先輩の研究コードについて\&講習会についてお話をした.
\end{itembox}

\begin{itembox}[r]{May/1}
輪講会でりょうくんと一緒に説明した.\\
SA-AMG法について知るために,反復解法部ではどのような手法を用いているのか確認した.
\end{itembox}

\begin{itembox}[r]{May/2}
午前中は,ヤコビ法とガウス・ザイデル法について,行列とベクトルの観点から理解した.\\
研究室に人が来ると,集中力がなくなることが分かった.\\
図書館が開いていない日は,用事がなければ学校に来る必要はなさそうであると考えた.
\end{itembox}

\begin{itembox}[r]{Mar/3}
CG法について,行列とベクトルの観点から理解した.\\
自転車で多摩川まで行った.清々しかった.\\
keyword:正定値対称行列
\end{itembox}

\section{column}
\begin{itembox}[c]{数学と学び方について}
	テキストは,分かる系より解ける系を選ぶ.最初は,コンパクトな本を使うと良い.これは,「理解は遅れてやってくる」ということによるものである.反復していくことで,頭を使わなくても解けるようになり,理解することの方により頭の資源の多くを割くことができる.\\
薄い \textless 読み \textgreater を塗り重ねる.\\
	挫折は, \textless 分からなくて苦しいから辞める \textgreater という流れからくるものである.とにかく最後まで行くことにする.最初は目次,その次は大見出しだけ最後まで読んで行く,といった感じである.その時点の自分の実力で分かるところだけ拾って読む.最初は,「どこなら分かりそうか」を探すだけでも良い.
\end{itembox}

\begin{itembox}[c]{LaTex tips}
texファイルをvimで編集するときの設定\\
自動インデントをしないようにする.$\rightarrow$ \textasciitilde/.vim/indent/tex.vimを以下のように記述して作成する.\\
let b:did\_indent = 1
\end{itembox}

\end{document}


\begin{comment}
	これは,プログラムの \ref{bobobobo}
\lstinputlisting[caption= ぼぼぼぼ, label= bobobobo]{dotTorF.txt}
	\title{report}
	\author{J114081 辻祥太}
	\data{}
	\maketitle
	\newpage

	\tableofcontents %Making Table of Contents
	\newpage

	\begin{itemize}
	\item
	\end{itemize}

	\begin{table}[H]
	\caption{•}
	\label{table:}
	\begin{center}
	\begin{tabular}{lrr}
	&&
	\end{tabular}
	\end{center}
	\end{table}

	\begin{figure}[H]
	\begin{center}
	\includegraphics[width=cm]{.png}
	\caption{}
	\label{fig:}
	\end{center}
	\end{figure}

	\begin{figure}[H]
	\begin{minipage}{0.5\columnwidth}
		\begin{center}
		\includegraphics[width=\columnwidth]{.png}
		\caption\fi{•}
		\label{•}
		\end{center}
	\end{minipage}
	\begin{minipage}{0.5\columnwidth}
		\begin{center}
		\includegraphics[width=\columnwidth]{.png}
		\caption{•}
		\label{•}
		\end{center}
	\end{minipage}
	\end{figure}
\end{comment}
