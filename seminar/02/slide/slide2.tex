\documentclass[dvipdfmx]{beamer}
\usepackage{etex}
%\documentclass[dvipdfmx,usenames]{beamer}
\usepackage{graphicx}
\usepackage{color}
\usepackage{listings}
\usetheme{Madrid}
\useinnertheme{circles}
\useoutertheme{default}
%\usepackage{beamerthemeshadow}
\usepackage{tikz}
\usepackage[varg]{txfonts}
\usepackage{booktabs}
\renewcommand{\kanjifamilydefault}{\gtdefault}
\setbeamercovered{transparent}
\setbeamertemplate{navigation symbols}{}
\everymath{\displaystyle}
\AtBeginSection[]
{\begin{frame}{発表の流れ}
	\begin{multicols}{2}[]
  \tableofcontents[currentsection]
	\end{multicols}
\end{frame}}
%ページ番号の表示
%\setbeamertemplate{footline}[frame number]
\usepackage{hyperref}
%しおりをつくるsectionの深さや,目次のリンクの色などを指定
%\usepackage[bookmarksopenlevel=2]{hyperref}
\usepackage{pxjahyper}
%amsmathとhyperrefとの互換性
\usepackage{amsmath}
\let\equation\gather
\let\endequation\endgather

\setbeamertemplate{enumerate item}[circle]

\newcommand{\backupbegin}{
  \newcounter{framenumberappendix}
  \setcounter{framenumberappendix}{\value{framenumber}}
}
\newcommand{\backupend}{
  \addtocounter{framenumberappendix}{-\value{framenumber}}
  \addtocounter{framenumber}{\value{framenumberappendix}}
}

%\lstset{
%	language= C,
%	numbers=left,		numbersep= 10pt,		numberstyle= \ttfamily,
%	breaklines=ture,	breakindent= 20pt,
%	%frame= shadowbox,	
%	framesep= 3pt,	rulesep= 4pt,
%	stepnumber= 1,
%	backgroundcolor= \color{white},
%	rulesepcolor= \color{cyan},
%	basicstyle= \ttfamily, %columns= [1][fullflexible],
%	tabsize=2,
%	label= %
%}

\lstdefinestyle{customc}{
	belowcaptionskip=1\baselineskip,
	breaklines=true,
	frame=L,
	xleftmargin=\parindent,
	language=C,
	showstringspaces=false,
	basicstyle=\footnotesize\ttfamily,
	%basicstyle=\tiny\ttfamily,
	keywordstyle=\bfseries\color{green!40!black},
	commentstyle=\itshape\color{purple!40!black},
	identifierstyle=\color{blue},
	stringstyle=\color{orange},
	tabsize=2,
}

%\lstdefinestyle{customasm}{
%	belowcaptionskip=1\baselineskip,
%	frame=L,
%	xleftmargin=\parindent,
%	language=[x86masm]Assembler,
%	basicstyle=\footnotesize\ttfamily,
%	commentstyle=\itshape\color{purple!40!black},
%}
\lstset{escapechar=@,style=customc}


\usepackage{subcaption}
\usepackage{multicol}
\usepackage{lmodern}
\usepackage{advdate}

\title[seminar 3rd]{学部ゼミ 第3回}
\institute{Kogakuin University}
\author[Shota Tsuji]{情報学部コンピュータ科 j114081 辻 祥太}
%\date{\AdvanceDate[1]\today}
\date{\today}


\begin{document}
\begin{frame}
  \titlepage 
\end{frame} 
%\begin{frame}[plain]
%  \frametitle{Contents}
%  \tableofcontents
%\end{frame}

\section*{実行環境}
\begin{frame}
	\frametitle{Aluminium(Intel Xeon X5570)}
	\begin{columns}
	\begin{column}{0.48\textwidth}
	\begin{itemize}
		\item Frequency 2.93[GHz]
		\item cache\_size 8192 KB
		\item 1Clockあたりの演算回数\\ 2[Flops/Clock]
	\end{itemize}

	\begin{itemize}
		\item \$ cat /proc/cpuinfo で確認できる
	\end{itemize}
	\end{column}

	\end{columns}
\end{frame}

\section{性能値の計算}
\begin{frame}
  \frametitle{性能値の計算}
	\begin{block}{理論性能値}
	\begin{equation*}
		%(\mbox{性能値GFlops}) = (\mbox{1Clockあたりの演算回数}) * (\mbox{Frequency}) * (\mbox{コア数}) \\
		(GFlops) = (FLOPS/Clock) * (Frequency) * (Num\_Cores)
  \end{equation*}
	\end{block}

	\begin{block}{実測性能値}
	\begin{center}
  \begin{equation*}
		(\mbox{GFlops}) = \frac{(\mbox{計算1回あたりの演算回数})*(\mbox{計算回数})}{(\mbox{全体で掛かった時間)*($10^9$)}}
  \end{equation*}
	\end{center}
	\end{block}
	Flopsとは,1秒あたりの演算回数(Flop/sec)\\
	Aluminium計算機での理論値
	\begin{equation*}
		2 * 2.93 * 1 = 5.86GFlops
  \end{equation*}
\end{frame}


\section{行列行列積}
\begin{frame}
  \frametitle{行列行列積とチューニング手法}
  \begin{itemize}
		\item N*Nの正方行列の積$(100\leq N \leq 1000)$ \\ 今回は刻み幅を50にした
		\item ループ交換
		\item 転置
		\item ブロッキング
	\end{itemize}
\end{frame}

% Loop Exchange
\section{ループ交換}
\begin{frame}
\frametitle{ループ交換(単純なijkループ)}
\lstinputlisting[label=ijk]{./ijk.c}
\begin{tikzpicture}[baseline=40pt]
	\draw [] (0,0) grid (3,3);
	\draw (0,3) node{}+(0,0) rectangle ++(1,-1)[fill=pink,opacity=0.5];
\end{tikzpicture}
+=
\begin{tikzpicture}[baseline=40pt]
	\draw [] (0,0) grid (3,3);
	\draw (0,3) node{}+(0,0) rectangle ++(1,-1)[fill=cyan,];
\end{tikzpicture}
*
\begin{tikzpicture}[baseline=40pt]
	\draw [] (0,0) grid (3,3);
	\draw (0,3) node{}+(0,0) rectangle ++(1,-1)[fill=magenta,];
\end{tikzpicture}

\end{frame}
\begin{frame}
\frametitle{ループ交換(単純なijkループ)}
\lstinputlisting{./ijk.c}
\begin{tikzpicture}[baseline=40pt]
	\draw [] (0,0) grid (3,3);
	\draw (0,3) node{}+(0,0) rectangle ++(1,-1)[fill=pink,opacity=0.5];
\end{tikzpicture}
+=
\begin{tikzpicture}[baseline=40pt]
	\draw [] (0,0) grid (3,3);
	\draw (0,3) node{}+(1,0) rectangle ++(2,-1)[fill=cyan,];
\end{tikzpicture}
*
\begin{tikzpicture}[baseline=40pt]
	\draw [] (0,0) grid (3,3);
	\draw (0,3) node{}+(0,-1) rectangle ++(1,-2)[fill=magenta,];
\end{tikzpicture}

\end{frame}
\begin{frame}
\frametitle{ループ交換(単純なijkループ)}
\lstinputlisting{./ijk.c}
\begin{tikzpicture}[baseline=40pt]
	\draw [] (0,0) grid (3,3);
	\draw (0,3) node{}+(0,0) rectangle ++(1,-1)[fill=pink,opacity=0.5];
\end{tikzpicture}
+=
\begin{tikzpicture}[baseline=40pt]
	\draw [] (0,0) grid (3,3);
	\draw (0,3) node{}+(2,0) rectangle ++(3,-1)[fill=cyan,];
\end{tikzpicture}
*
\begin{tikzpicture}[baseline=40pt]
	\draw [] (0,0) grid (3,3);
	\draw (0,3) node{}+(0,-2) rectangle ++(1,-3)[fill=magenta,];
\end{tikzpicture}

\end{frame}
\begin{frame}
\frametitle{ループ交換(単純なijkループ)}
\begin{itemize}
	\item 内積形式\\これは,行列AとBの部分的な内積
\end{itemize}
\begin{tikzpicture}[baseline=40pt]
	\draw [] (0,0) grid (3,3);
	\draw (0,3) node{}+(0,0) rectangle ++(1,-1)[fill=pink,opacity=0.5];
\end{tikzpicture}
+=
\begin{tikzpicture}[baseline=40pt]
	\draw [] (0,0) grid (3,3);
	\draw (0,3) node{}+(0,0) rectangle ++(3,-1)[fill=cyan,];
\end{tikzpicture}
*
\begin{tikzpicture}[baseline=40pt]
	\draw [] (0,0) grid (3,3);
	\draw (0,3) node{}+(0,0) rectangle ++(1,-3)[fill=magenta,];
\end{tikzpicture}
\end{frame}

\begin{frame}
	\frametitle{ループ交換の種類}
	大きく分けて以下の3種類
	\begin{enumerate}
		\item 内積形式\\ {\tt (i,j,k)} と {\tt (j,i,k)}
		\item 外積形式\\ {\tt (k,i,j)} と {\tt (k,j,i)}
		\item 中間積形式\\ {\tt (i,k,j)} と {\tt (j,k,i)}
	\end{enumerate}
\end{frame}


\begin{frame}
	\frametitle{ループ交換(外積形式)}
	\lstinputlisting[label=kji]{./kji.c}
	\begin{tikzpicture}[baseline=40pt]
		\draw [] (0,0) grid (3,3);
		\draw (0,3) node{}+(0,0) rectangle ++(1,-1)[fill=pink,opacity=0.5];
	\end{tikzpicture}
	+=
	\begin{tikzpicture}[baseline=40pt]
		\draw [] (0,0) grid (3,3);
		\draw (0,3) node{}+(0,0) rectangle ++(1,-1)[fill=cyan,];
	\end{tikzpicture}
	*
	\begin{tikzpicture}[baseline=40pt]
		\draw [] (0,0) grid (3,3);
		\draw (0,3) node{}+(0,0) rectangle ++(1,-1)[fill=magenta,];
	\end{tikzpicture}
\end{frame}


\begin{frame}
	\frametitle{ループ交換(外積形式)}
	\lstinputlisting{./kji.c}
	\begin{tikzpicture}[baseline=40pt]
		\draw [] (0,0) grid (3,3);
		\draw (0,0) node{}+(0,1) rectangle ++(1,2)[fill=pink,opacity=0.5];
	\end{tikzpicture}
	+=
	\begin{tikzpicture}[baseline=40pt]
		\draw [] (0,0) grid (3,3);
		\draw (0,0) node{}+(0,1) rectangle ++(1,2)[fill=cyan];	
	\end{tikzpicture}
	*
	\begin{tikzpicture}[baseline=40pt]
		\draw [] (0,0) grid (3,3);
		\draw (0,3) node{}+(0,0) rectangle ++(1,-1)[fill=magenta,];
	\end{tikzpicture}
\end{frame}


\begin{frame}
	\frametitle{ループ交換(外積形式)}
	\lstinputlisting{./kji.c}
	\begin{tikzpicture}[baseline=40pt]
		\draw [] (0,0) grid (3,3);
		\draw (0,3) node{}+(0,-2) rectangle ++(1,-3)[fill=pink,opacity=0.5];
	\end{tikzpicture}
	+=
	\begin{tikzpicture}[baseline=40pt]
		\draw [] (0,0) grid (3,3);
		\draw (0,3) node{}+(0,-2) rectangle ++(1,-3)[fill=cyan,];
	\end{tikzpicture}
	*
	\begin{tikzpicture}[baseline=40pt]
		\draw [] (0,0) grid (3,3);
		\draw (0,3) node{}+(0,0) rectangle ++(1,-1)[fill=magenta,];
	\end{tikzpicture}
\end{frame}


\begin{frame}
	\frametitle{ループ交換(外積形式)}
	\begin{itemize}
		\item 外積形式\\これは,行列AとBの部分的な外積
	\end{itemize}
	\begin{tikzpicture}[baseline=40pt]
		\draw [] (0,0) grid (3,3);
		\draw (0,0) node{}+(0,0) rectangle ++(1,3)[fill=pink,opacity=0.5];
	\end{tikzpicture}
	+=
	\begin{tikzpicture}[baseline=40pt]
		\draw [] (0,0) grid (3,3);
		\draw (0,0) node{}+(0,0) rectangle ++(1,3)[fill=cyan,];
	\end{tikzpicture}
	*
	\begin{tikzpicture}[baseline=40pt]
		\draw [] (0,0) grid (3,3);
		\draw (0,0) node{}+(0,2) rectangle ++(3,3)[fill=magenta,];
	\end{tikzpicture}
\end{frame}


\begin{frame}
	\frametitle{ループ交換(中間積形式)}
	\lstinputlisting{./jki.c}
	\begin{tikzpicture}[baseline=40pt]
		\draw [] (0,0) grid (3,3);
		\draw (0,0) node{}+(0,2) rectangle ++(1,3)[fill=pink,opacity=0.5];
	\end{tikzpicture}
	+=
	\begin{tikzpicture}[baseline=40pt]
		\draw [] (0,0) grid (3,3);
		\draw (0,0) node{}+(0,2) rectangle ++(1,3)[fill=cyan,];
	\end{tikzpicture}
	*
	\begin{tikzpicture}[baseline=40pt]
		\draw [] (0,0) grid (3,3);
		\draw (0,0) node{}+(0,2) rectangle ++(1,3)[fill=magenta,];
	\end{tikzpicture}
\end{frame}

\begin{frame}
	\frametitle{ループ交換(中間積形式)}
	\lstinputlisting{./jki.c}
	\begin{tikzpicture}[baseline=40pt]
		\draw [] (0,0) grid (3,3);
		\draw (0,0) node{}+(0,1) rectangle ++(1,2)[fill=pink,opacity=0.5];
	\end{tikzpicture}
	+=
	\begin{tikzpicture}[baseline=40pt]
		\draw [] (0,0) grid (3,3);
		\draw (0,0) node{}+(0,1) rectangle ++(1,2)[fill=cyan,];
	\end{tikzpicture}
	*
	\begin{tikzpicture}[baseline=40pt]
		\draw [] (0,0) grid (3,3);
		\draw (0,0) node{}+(0,1) rectangle ++(1,2)[fill=magenta,];
	\end{tikzpicture}
\end{frame}

\begin{frame}
	\frametitle{ループ交換(中間積形式)}
	\lstinputlisting{./jki.c}
	\begin{tikzpicture}[baseline=40pt]
		\draw [] (0,0) grid (3,3);
		\draw (0,0) node{}+(0,0) rectangle ++(1,1)[fill=pink,opacity=0.5];
	\end{tikzpicture}
	+=
	\begin{tikzpicture}[baseline=40pt]
		\draw [] (0,0) grid (3,3);
		\draw (0,0) node{}+(0,0) rectangle ++(1,1)[fill=cyan,];
	\end{tikzpicture}
	*
	\begin{tikzpicture}[baseline=40pt]
		\draw [] (0,0) grid (3,3);
		\draw (0,0) node{}+(0,0) rectangle ++(1,1)[fill=magenta,];
	\end{tikzpicture}
\end{frame}

\begin{frame}
	\frametitle{ループ交換(中間積形式)}
	\begin{itemize}
		\item これは,外積でも内積でもない
	\end{itemize}
	\begin{tikzpicture}[baseline=40pt]
		\draw [] (0,0) grid (3,3);
		\draw (0,0) node{}+(0,0) rectangle ++(1,3)[fill=pink,opacity=0.5];
	\end{tikzpicture}
	+=
	\begin{tikzpicture}[baseline=40pt]
		\draw [] (0,0) grid (3,3);
		\draw (0,0) node{}+(0,0) rectangle ++(1,3)[fill=cyan,];
	\end{tikzpicture}
	*
	\begin{tikzpicture}[baseline=40pt]
		\draw [] (0,0) grid (3,3);
		\draw (0,0) node{}+(0,0) rectangle ++(1,3)[fill=magenta,];
	\end{tikzpicture}
\end{frame}

\begin{frame}
	\frametitle{ループ交換6種の計測結果}
	\begin{figure}[h]
		\includegraphics[height=0.7\textheight]{./a.jpeg}
	\end{figure}
\end{frame}

\begin{frame}
	\frametitle{ループ交換の結果}
	\begin{itemize}
		\item 列方向アクセスが少ないものは,少し性能が出た
		\item 結果は,3種類になった
	\end{itemize}
\end{frame}



\section{ブロッキング}
\begin{frame}
\frametitle{単純なijkループ}
%\lstinputlisting{./ijk.c}
\begin{itemize}
	\item キャッシュ・ミスヒットが多発\\メモリに取りに行く\\キャッシュの追い出し
\end{itemize}
\begin{tikzpicture}[baseline=40pt, scale=3/4]
	\draw [] (0,0) grid (4,4);
	%\draw (0,0) node{}+(0,3) rectangle ++(1,4)[fill=pink,opacity=0.5];
	\draw (0,0) node{}+(0,3) rectangle ++(1,4)[fill=pink,];
	\draw (2,3.5) node{cache miss};
\end{tikzpicture}
+=
\begin{tikzpicture}[baseline=40pt, scale=3/4]
	\draw [] (0,0) grid (4,4);
	\draw (0,0) node{}+(0,3) rectangle ++(1,4)[fill=cyan,];
	\draw (2,3.5) node{cache miss};
\end{tikzpicture}
*
\begin{tikzpicture}[baseline=40pt, scale=3/4]
	\draw [] (0,0) grid (4,4);
	\draw (0,0) node{}+(0,3) rectangle ++(1,4)[fill=magenta,];
	\draw (2,3.5) node{cache miss};
\end{tikzpicture}

\end{frame}
\begin{frame}
\frametitle{単純なijkループ}
%\lstinputlisting{./ijk.c}
\begin{itemize}
	\item キャッシュ・ミスヒットが多発\\メモリに取りに行く\\キャッシュの追い出し
\end{itemize}
\begin{tikzpicture}[baseline=40pt, scale=3/4]
	\draw [] (0,0) grid (4,4);
	\draw (0,0) node{}+(0,3) rectangle ++(4,4)[fill=gray,opacity=0.5];
	\draw (0,0) node{}+(0,3) rectangle ++(1,4)[fill=pink,];
\end{tikzpicture}
+=
\begin{tikzpicture}[baseline=40pt, scale=3/4]
	\draw [] (0,0) grid (4,4);
	\draw (0,0) node{}+(0,3) rectangle ++(4,4)[fill=gray,opacity=0.5];
	\draw (0,0) node{}+(1,3) rectangle ++(2,4)[fill=cyan,];
\end{tikzpicture}
*
\begin{tikzpicture}[baseline=40pt, scale=3/4]
	\draw [] (0,0) grid (4,4);
	\draw (0,0) node{}+(0,3) rectangle ++(4,4)[fill=gray,opacity=0.5];
	\draw (0,0) node{}+(0,2) rectangle ++(1,3)[fill=magenta,];
	\draw (2,2.5) node{cache miss};
\end{tikzpicture}
\end{frame}


\begin{frame}
\frametitle{単純なijkループ}
%\lstinputlisting{./ijk.c}
\begin{itemize}
	\item キャッシュ・ミスヒットが多発\\メモリに取りに行く\\キャッシュの追い出し
\end{itemize}
\begin{tikzpicture}[baseline=40pt, scale=3/4]
	\draw [] (0,0) grid (4,4);
	\draw (0,0) node{}+(0,3) rectangle ++(4,4)[fill=gray,opacity=0.5];
	\draw (0,0) node{}+(0,3) rectangle ++(1,4)[fill=pink,];
\end{tikzpicture}
+=
\begin{tikzpicture}[baseline=40pt, scale=3/4]
	\draw [] (0,0) grid (4,4);
	\draw (0,0) node{}+(0,3) rectangle ++(4,4)[fill=gray,opacity=0.5];
	\draw (0,0) node{}+(2,3) rectangle ++(3,4)[fill=cyan,];
\end{tikzpicture}
*
\begin{tikzpicture}[baseline=40pt, scale=3/4]
	\draw [] (0,0) grid (4,4);
	\draw (0,0) node{}+(0,3) rectangle ++(4,4)[fill=gray,opacity=0.5];
	\draw (0,0) node{}+(0,2) rectangle ++(4,3)[fill=gray,opacity=0.5];
	\draw (0,0) node{}+(0,1) rectangle ++(1,2)[fill=magenta,];
	\draw (2,1.5) node{cache miss};
\end{tikzpicture}

\end{frame}


\begin{frame}
\frametitle{単純なijkループ}
%\lstinputlisting{./ijk.c}
\begin{itemize}
	\item キャッシュ・ミスヒットが多発\\メモリに取りに行く\\キャッシュの追い出し
\end{itemize}
\begin{tikzpicture}[baseline=40pt, scale=3/4]
	\draw [] (0,0) grid (4,4);
	\draw (0,0) node{}+(0,3) rectangle ++(4,4)[fill=gray,opacity=0.5];
	\draw (0,0) node{}+(0,3) rectangle ++(1,4)[fill=pink,];
\end{tikzpicture}
+=
\begin{tikzpicture}[baseline=40pt, scale=3/4]
	\draw [] (0,0) grid (4,4);
	\draw (0,0) node{}+(0,3) rectangle ++(4,4)[fill=gray,opacity=0.5];
	\draw (0,0) node{}+(3,3) rectangle ++(4,4)[fill=cyan,];
\end{tikzpicture}
*
\begin{tikzpicture}[baseline=40pt, scale=3/4]
	\draw [] (0,0) grid (4,4);
	\draw (0,0) node{}+(0,3) rectangle ++(4,4)[fill=gray,opacity=0.5];
	\draw (0,0) node{}+(0,2) rectangle ++(4,3)[fill=gray,opacity=0.5];
	\draw (0,0) node{}+(0,1) rectangle ++(4,2)[fill=gray,opacity=0.5];
	\draw (0,0) node{}+(0,0) rectangle ++(1,1)[fill=magenta,];
	\draw (2,0.5) node{cache miss};
\end{tikzpicture}
\end{frame}


% Blocking
\begin{frame}
	\frametitle{ブロッキング}
	\begin{itemize}
		\item 取ってきた値を繰り返し参照する
		\item メモリへのアクセス減
	\end{itemize}

\begin{tikzpicture}[baseline=40pt, scale=3/4]
	\draw [] (0,0) grid (4,4);
	%\draw (0,0) node{}+(0,3) rectangle ++(2,4)[fill=gray,opacity=0.5];
	\draw (0,0) node{}+(0,3) rectangle ++(1,4)[fill=pink];
	\draw (2,3.5) node{cache miss};
\end{tikzpicture}
+=
\begin{tikzpicture}[baseline=40pt, scale=3/4]
	\draw [] (0,0) grid (4,4);
	\draw (0,0) node{}+(0,3) rectangle ++(1,4)[fill=cyan,];
	\draw (2,3.5) node{cache miss};
\end{tikzpicture}
*
\begin{tikzpicture}[baseline=40pt, scale=3/4]
	\draw [] (0,0) grid (4,4);
	\draw (0,0) node{}+(0,3) rectangle ++(1,4)[fill=magenta,];
	\draw (2,3.5) node{cache miss};
\end{tikzpicture}
\end{frame}


\begin{frame}
	\frametitle{ブロッキング}
	\begin{itemize}
		\item 取ってきた値を繰り返し参照する
		\item メモリへのアクセス減
	\end{itemize}

\begin{tikzpicture}[baseline=40pt, scale=3/4]
	\draw [] (0,0) grid (4,4);
	\draw (0,0) node{}+(0,3) rectangle ++(2,4)[fill=gray,opacity=0.5];
	\draw (0,0) node{}+(0,3) rectangle ++(1,4)[fill=pink];
	%\draw (2,3.5) node{cache miss};
\end{tikzpicture}
+=
\begin{tikzpicture}[baseline=40pt, scale=3/4]
	\draw [] (0,0) grid (4,4);
	\draw (0,0) node{}+(0,3) rectangle ++(2,4)[fill=gray,opacity=0.5];
	\draw (0,0) node{}+(1,3) rectangle ++(2,4)[fill=cyan,];
	%\draw (2,3.5) node{cache miss};
\end{tikzpicture}
*
\begin{tikzpicture}[baseline=40pt, scale=3/4]
	\draw [] (0,0) grid (4,4);
	\draw (0,0) node{}+(0,3) rectangle ++(2,4)[fill=gray,opacity=0.5];
	\draw (0,0) node{}+(1,3) rectangle ++(2,4)[fill=magenta,];
	%\draw (2,3.5) node{cache miss};
\end{tikzpicture}
\end{frame}


\begin{frame}
	\frametitle{ブロッキング}
	\begin{itemize}
		\item 取ってきた値を繰り返し参照する
		\item メモリへのアクセス減
	\end{itemize}

\begin{tikzpicture}[baseline=40pt, scale=3/4]
	\draw [] (0,0) grid (4,4);
	\draw (0,0) node{}+(0,3) rectangle ++(2,4)[fill=gray,opacity=0.5];
	\draw (0,0) node{}+(1,3) rectangle ++(2,4)[fill=pink];
	%\draw (2,3.5) node{cache miss};
\end{tikzpicture}
+=
\begin{tikzpicture}[baseline=40pt, scale=3/4]
	\draw [] (0,0) grid (4,4);
	\draw (0,0) node{}+(0,3) rectangle ++(2,4)[fill=gray,opacity=0.5];
	\draw (0,0) node{}+(0,3) rectangle ++(1,4)[fill=cyan,];
	%\draw (2,3.5) node{cache miss};
\end{tikzpicture}
*
\begin{tikzpicture}[baseline=40pt, scale=3/4]
	\draw [] (0,0) grid (4,4);
	\draw (0,0) node{}+(0,3) rectangle ++(2,4)[fill=gray,opacity=0.5];
	\draw (0,0) node{}+(0,2) rectangle ++(1,3)[fill=magenta,];
	\draw (2,2.5) node{cache miss};
\end{tikzpicture}
\end{frame}


\begin{frame}
	\frametitle{ブロッキング}
	\begin{itemize}
		\item 取ってきた値を繰り返し参照する
		\item メモリへのアクセス減
	\end{itemize}

\begin{tikzpicture}[baseline=40pt, scale=3/4]
	\draw [] (0,0) grid (4,4);
	\draw (0,0) node{}+(0,3) rectangle ++(2,4)[fill=gray,opacity=0.5];
	\draw (0,0) node{}+(1,3) rectangle ++(2,4)[fill=pink];
	%\draw (2,3.5) node{cache miss};
\end{tikzpicture}
+=
\begin{tikzpicture}[baseline=40pt, scale=3/4]
	\draw [] (0,0) grid (4,4);
	\draw (0,0) node{}+(0,3) rectangle ++(2,4)[fill=gray,opacity=0.5];
	\draw (0,0) node{}+(1,3) rectangle ++(2,4)[fill=cyan,];
	%\draw (2,3.5) node{cache miss};
\end{tikzpicture}
*
\begin{tikzpicture}[baseline=40pt, scale=3/4]
	\draw [] (0,0) grid (4,4);
	\draw (0,0) node{}+(0,3) rectangle ++(2,4)[fill=gray,opacity=0.5];
	\draw (0,0) node{}+(0,2) rectangle ++(2,3)[fill=gray,opacity=0.5];
	\draw (0,0) node{}+(1,2) rectangle ++(2,3)[fill=magenta,];
	%\draw (2,2.5) node{cache miss};
\end{tikzpicture}
\end{frame}


\begin{frame}
	\frametitle{ブロッキング(Code)}
	\begin{columns}
	\begin{column}{0.60\textwidth}
	\lstinputlisting{./b.c}
	\end{column}
	\begin{column}{0.30\textwidth}
		\begin{itemize}
			\item nは行列サイズ
			\item ブロックサイズはその半分
		\end{itemize}
	\end{column}
	\end{columns}
\end{frame}

\begin{frame}
	\frametitle{ブロッキング-Aluminium計算機のキャッシュサイズ}
	\begin{itemize}
		\item chash\_size 8192KB/core
		\item chash\_alignment 64Byte
		\item 行列サイズN=1000のとき,ブロック3個が収まるようにする
	\end{itemize}
	\begin{equation*}
		8 * 500 * 500 * 3 = 6000KB \leq 8192KB
	\end{equation*}
\end{frame}


\begin{frame}
	\frametitle{ブロッキングの結果}

	\begin{figure}[h]
		\includegraphics[height=0.7\textheight]{./b.jpeg}
	\end{figure}
\end{frame}

\begin{frame}
	\frametitle{ブロッキングの結果}
	\begin{itemize}
		\item 全体的にみて,ブロッキングの方が少し性能が出た
		\item 局所的に単純ijkループに劣る
	\end{itemize}
\end{frame}
%\section{転置}

%\section{実行とコード内容}
%\begin{frame}
%	\frametitle{Execution}
%	\begin{columns}
%	\begin{column}{0.60\textwidth}
%	Command
%	Content of script
%	\end{column}
%	\begin{column}{0.30\textwidth}
%	\begin{itemize}
%		\item shell scriptで行列の大きさNを変えながら回す\\
%		\item 刻み幅は50\\
%		\item result???.csvに書き込む
%	\end{itemize}
%	\end{column}
%	\end{columns}
%\end{frame}
%
%\begin{frame}
%	\frametitle{Sorce Code of sem\_mat\_mat.c}
%	\begin{columns}
%	\begin{column}{0.60\textwidth}
%		時間計測
%	\end{column}
%	\begin{column}{0.30\textwidth}
%		omp\_get\_wtime()を使う
%		行列Cへの値を足しこむ部分を囲む
%	\end{column}
%	\end{columns}
%\end{frame}
%
%
%\begin{frame}
%	\frametitle{考察}
%	愚直な行列行列積(2次元配列の場合)
%	\begin{itemize}
%		\item 行列bに対し,列方向でアクセスしている\\ 空間的局所性を考えていないプログラム
%	\end{itemize}
%	%double型のサイズは,8byte
%\end{frame}
%
%\begin{frame}
%	\frametitle{これからすること}
%	\begin{itemize}
%	\item 行列を1次元でとる
%	\item 最適化オプション試してみる
%	\end{itemize}
%\end{frame}

% codeのサイズを小さめにする
% 式の番号表示しない
% 1次元配列にしたときの計算
% サイズ/timeの結果
% 理論性能との比較
% コンパイルオプションつき
% adressen あドレッセん
% よこせん(スライド一枚を分割)
\appendix
\backupbegin

\begin{frame}
  \begin{center}
    \huge appendixes
  \end{center}
\end{frame}

\begin{frame}
  \frametitle{使用した計算機の特徴と実行環境}
	\begin{columns}
	\begin{column}{0.48\textwidth}
	\begin{itemize}
			\item コンパイル\\ gcc -fopenmp sem\_mat\_mat.c -lm
			\item コアは1個のみ使用
	\end{itemize}
	\end{column}

	\begin{column}{0.48\textwidth}
  \begin{table}
    \caption{Aluminium計算機}
		\centering
    \begin{tabular}{ c | r } \toprule
			特徴 & 内容\\ \midrule
      種類 & Intel Xeon X5570\\
      コア数 & 2コアまで\\
			周波数 & 2.93GHz\\ \bottomrule
    \end{tabular}
  \end{table}
	\end{column}
	\end{columns}
\end{frame}


\begin{frame}
%  \frametitle{参考文献}
%  \begin{thebibliography}{9}
%    \bibitem{pca_start}
%    統計学研究所 www.statistics.co.jp/reference/software\_R/statR\_9\_principal.pdf
%    \bibitem{knn}
%    http://machinelearningmastery.com/tutorial-to-implement-k-nearest-neighbors-in-python-from-scratch/
%  \end{thebibliography}
\end{frame}

\backupend
\end{document}
