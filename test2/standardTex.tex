\documentclass[10pt, a4paper]{jsarticle}
\usepackage[dvipdfmx, hiresbb]{graphicx} %Bring Images[Driver, Exactly]
\usepackage{here}
\usepackage{comment}
\usepackage{array, booktabs} %Setting for Table
\usepackage{float} %Locating Figure and Table
\usepackage[top=20truemm, bottom=25truemm, left=15truemm, right=15truemm]{geometry} %Setting Page Balance
\usepackage{listings} %For Showing Source Code
\usepackage{xcolor} %Using Color ->Source Code
\usepackage{amsmath}
\renewcommand{\lstlistingname}{ソースコード} %Caption for Soure Code
\lstset{
	language= C,
	numbers= left,		numbersep= 10pt,		numberstyle= \ttfamily,
	breaklines= ture,	breakindent= 20pt,
	frame= shadowbox,	framesep= 3pt,	rulesep= 4pt,
	stepnumber= 1,
	backgroundcolor= \color{white},
	rulesepcolor= \color{cyan},
	basicstyle= \ttfamily, columns= [1][fullflexible],
	tabsize= 4,
	label= }
%End of Priamble


\begin{document}
\tableofcontents %Making Table of Contents
\newpage

\section{レポート内容}
今回は,複数の非線形モデルに対する,ある初期値での個体数x,yの時間ステップでの変化の様子とそれに対応した相平面の描画を行なった.用いた非線形モデルは,以下の3種類である.

	\begin{equation}
	\begin{cases}
		\dot{x} = x \left(3-x-y \right) \\
		\dot{y} = y \left(2-x-y \right)
	\end{cases}
	\end{equation}
	\begin{equation}
		A = \begin{pmatrix}
			-2x+3-y & -x \\
			-y & -2y+2-x
			\end{pmatrix}
	\end{equation}
	\begin{equation}
	\begin{cases}
		\dot{x} = x \left(3-2x-y \right) \\
		\dot{y} = y \left(2-x-y \right)
	\end{cases}
	\end{equation}
	\begin{equation}
		A = \begin{pmatrix}
			-4x+3-y & -x \\
			-y & -2y+2-x
			\end{pmatrix}
	\end{equation}
	\begin{equation}
	\begin{cases}
		\dot{x} = x \left(3-2x-2y \right) \\
		\dot{y} = y \left(2-x-y \right)
	\end{cases}
	\end{equation}
	\begin{equation}
		A = \begin{pmatrix}
			-4+3-y & -2x \\
			-y & -2y+2-x
			\end{pmatrix}
	\end{equation}




\section{使用器具等} 
\section{実験} 
\section{結果}
\section{考察}
\section{結論}
\section{参考文献}

\end{document}


\begin{comment}
	これは,プログラムの \ref{bobobobo}
\lstinputlisting[caption= ぼぼぼぼ, label= bobobobo]{dotTorF.txt}
	\title{report}
	\author{J114081 辻祥太}
	\data{}
	\maketitle
	\newpage

	\tableofcontents %Making Table of Contents
	\newpage
	
	\begin{itemize}
	\item
	\end{itemize}
	
	\begin{table}[H]
	\caption{•}
	\label{table:}
	\begin{center}
	\begin{tabular}{lrr}
	&&
	\end{tabular}
	\end{center}
	\end{table}
	
	\begin{figure}[H]
	\begin{center}
	\includegraphics[width=cm]{.png}
	\caption{}
	\label{fig:}
	\end{center}
	\end{figure}
	
	\begin{figure}[H]
	\begin{minipage}{0.5\columnwidth}
		\begin{center}
		\includegraphics[width=\columnwidth]{.png}
		\caption\fi{•}
		\label{•}
		\end{center}
	\end{minipage}
	\begin{minipage}{0.5\columnwidth}
		\begin{center}
		\includegraphics[width=\columnwidth]{.png}
		\caption{•}
		\label{•}
		\end{center}
	\end{minipage}
	\end{figure}

	% 数式のための環境
	\begin{equation}
	\end{equation}
\end{comment}
